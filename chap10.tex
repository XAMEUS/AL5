\chapter{All pairs shortest paths}

\textit{"Plus courts chemins toutes sources toutes destinations"}

Étant donné un graphe $G$ et une fonction de coût $w$, on veut trouver $\forall u,\, v \in S$ un plus court chemin de $u$ à $v$.

\section{Dijkstra, plusieurs fois}
Lancer Dijkstra sur tous les sommets ; complexité en temps : $n \times (m + n \log n)$ ($n$ sommets, $m$ arêtes).

\section{Modification de Bellman-Ford}
\textbf{\textcolor{ocre}{Idée}} : Au temps $t$, calculer $\delta^{t}(u\, v)$ en utilisant $\delta^{t-1}(u\, v) \ \forall u,\, v$.

Initialement $\delta^{1}(u\, v) = w(u,\, v) \ \forall u,\, v$

$\forall t$, calculer $\delta^{t+1}$ en fonction de $\delta^{t}$ ?

On a pour $u,\, v$, un plus court chemin de longueur au plus $t$.

Pour obtenir un plus court chemin de $u$ à $v$ en passant par au plus $t + 1$ arêtes, il suffit de prendre $\min \lbrace \delta^{t}(u\, v),\, \min \lbrace \delta^{t}(u\, x) + w(x,\, v) \rbrace \rbrace$ $x$ voisins de $v$.

On peut s'arrêter lorsque $t = n-1$ (après on forme des cycles).

On note \texttt{Pred[u][v]} le prédécesseur de $v$ dans le plus court chemin de $u$ à $v$.

\begin{figure}[h]
	\centering
	\begin{tikzpicture}
		\GraphInit[vstyle=Normal]
		\tikzset{VertexStyle/.append  style={fill = node}}
		\tikzset{VertexStyle/.append  style={text = white}}
		\SetGraphUnit{1.5}
		\Vertex{A}
		\NOEA(A){B}
		\SOEA(A){F}
		\SetGraphUnit{2.1}
		\EA(B){C}
		\EA(F){E}
		\SetGraphUnit{1.5}
		\NOEA(E){D}
		\Edge[label=$20$](A)(B)
		\Edge[label=$2$](A)(F)
		\Edge[label=$40$](F)(C)
		\Edge[label=$2$](B)(C)
		\Edge[label=$2$](F)(E)
		\Edge[label=$2$](E)(D)
		\Edge[label=$2$](C)(D)
	\end{tikzpicture}
\end{figure}

\begin{figure}[h]
\centering
\begin{tabular}{c c c}
	& \texttt{D} & \texttt{Pred} \vspace*{0.2cm} \\
	$t = 1$ & 
	\begin{tabular}{c | c c c c c c}
		  & A & B & C & D & E & F \\ \hline
		A & 0 & 20 & - & - & - & 2 \\
		B & 20 & 0 & 2 & - & - & - \\
		C & - & 2 & 0 & 2 & - & 40 \\
		D & - & - & 2 & 0 & 2 & - \\
		E & - & - & - & 2 & 0 & 2 \\
		F & 2 & - & 40 & - & 2 & 0
	\end{tabular} &
	\begin{tabular}{c | c c c c c c}
		  & A & B & C & D & E & F \\ \hline
		A & A & A & - & - & - & A \\
		B & A & B & B & - & - & - \\
		C & - & B & C & C & - & C \\
		D & - & - & C & D & D & - \\
		E & - & - & - & D & E & E \\
		F & A & - & C & - & E & F \\
	\end{tabular} \\ & & \\
	$t = 2$ & 
	\begin{tabular}{c | c c c c c c}
		  & A & B & C & D & E & F \\ \hline
		A & 0 & 20 & \textcolor{ocre}{22} & - & - & 2 \\
		B & 20 & 0 & 2 & \textcolor{ocre}{4} & - & \textcolor{ocre}{22} \\
		C & \textcolor{ocre}{22} & 2 & 0 & 2 & \textcolor{ocre}{4} & 40 \\
		D & - & \textcolor{ocre}{4} & 2 & 0 & 2 & \textcolor{ocre}{4} \\
		E & - & - & \textcolor{ocre}{4} & 2 & 0 & 2 \\
		F & 2 & \textcolor{ocre}{22} & 40 & \textcolor{ocre}{4} & 2 & 0
	\end{tabular} &
	\begin{tabular}{c | c c c c c c}
		  & A & B & C & D & E & F \\ \hline
		A & A & A & \textcolor{ocre}{B} & - & - & A \\
		B & A & B & B & \textcolor{ocre}{C} & - & \textcolor{ocre}{A} \\
		C & \textcolor{ocre}{B} & B & C & C & \textcolor{ocre}{D} & C \\
		D & - & \textcolor{ocre}{C} & C & D & D & \textcolor{ocre}{E} \\
		E & - & - & \textcolor{ocre}{D} & D & E & E \\
		F & A & \textcolor{ocre}{A} & C & \textcolor{ocre}{E} & E & F \\
	\end{tabular} \\ etc. & &
\end{tabular}
\end{figure}

Pour trouver le plus court chemin de $C$ à $F$ il faut attendre $t = 3$, de $A à B$, $t=5 = n-1$ .

\vspace*{0.2cm}

\subsection*{Implémentation}
Dans l'implémentation de l'algorithme il va falloir faire des copies, sinon à l'instant $t$, \texttt{D} contient un mélange de  $\delta^{t}$ et $\delta^{t+1}$.

\lstinputlisting{Scripts/PCC1.py}

\section{En doublant la longueur des chemins à chaque itération}

\textbf{\textcolor{ocre}{Idée}} : Au temps $t$, calculer $\delta^{2^{t}}(u,\, v) \ \forall u,\, v$ en utilisant $\delta^{2^{t-1}}(u,\, v) \ \forall u,\, v$ .

Initialement :  $\delta^{2^{t}}(u,\, v) = w(u,\, v) \ \forall u,\, v$

$\forall t$ , calculer $\delta^{2^{t+1}}$ en fonction de $\delta^{2^{t+1}}$ ?

$\forall u$ , un plus court chemin de longueur au plus $2^{t+1} = 2 \times 2^{t}$ arêtes est composé de 2 plus courts chemins de longueur au plus $2^{t}$ .

\begin{figure}[h]
\centering
$u \overset{\leq \; 2^{t}}{\xrightarrow{\hspace*{2cm}}} x \overset{\leq \; 2^{t}}{\xrightarrow{\hspace*{2cm}}} v$
\end{figure}

Pour obtenir un plus court chemin de $u$ à $v$ en passant par au plus $2^{t+1}$ arêtes il suffit de prendre :

\begin{figure}[h]
\centering
$\min \lbrace \delta^{2^{t}}(u,\, x),\, \min\limits_{x \in S} \lbrace \delta^{2^{t}}(u,\, x) + \delta^{2^{t}}(x,\, v) \rbrace \rbrace$
\end{figure}

On peut s'arrêter lorsque $\delta^{2^{t}} \geq n - 1$ .

\subsection*{Implémentation}
On note \texttt{Pred[u][v]} le prédécesseur de $v$ dans le plus court chemin de $u$ à $v$.

\lstinputlisting{Scripts/PCC2.py}

\section{Algorithme de Floyd-Warshall}\index{Floyd-Warshall (Algorithme)}

\textbf{\textcolor{ocre}{Idée}} : Supposons que les sommets soient numérotés de $1$ à $n$.

À chaque itération $t$, on va calculer la longueur des plus courts chemins de $u$ à $v$ $\forall u,\, v \in S$ qui passent uniquement par les sommets $\lbrace 1,\, ... ,\, t\rbrace$

Appelons $\underset{R\subseteq S}{\delta^{R}}(u,\, x)$ la longueur du plus court chemin de $u$ à $v$ passant uniquement par des sommets appartenant à $R$.


De façon générale, si on a calculé $\forall u,\, v \ \ \delta^{1,\,...\,,\,k}(u,\, v)$

\begin{figure}[h]
\centering
$\delta^{1,\,...\,,\,k}(u,\, v) = \min \lbrace \delta^{1,\,...,\,k}(u,\, v),\, \delta^{1,\,...\, ,\,k}(u,\, k+1)\delta^{1,\,...\, ,\,k}(k+1,\, v)\rbrace$
\end{figure}

\newpage

\begin{figure}[h]
\centering
\begin{minipage}{.5\textwidth}
	\centering
	\begin{tabular}{c | c c c c c c}
		  & A & B & C & D & E & F \\ \hline
		A & 0 & 20 & - & - & - & 2 \\
		B & 20 & 0 & 2 & - & - & - \\
		C & - & 2 & 0 & 2 & - & 40 \\
		D & 40 & - & 2 & 0 & 2 & - \\
		E & - & - & - & 2 & 0 & 2 \\
		F & 2 & - & - & - & 2 & 0
	\end{tabular}
\end{minipage}%
\begin{minipage}{.5\textwidth}
	\centering
	\begin{tabular}{c | c c c c c c}
		  & A & B & C & D & E & F \\ \hline
		A & - & - & - & - & - & - \\
		B & - & - & - & - & - & - \\
		C & - & - & - & - & - & - \\
		D & - & - & - & - & - & - \\
		E & - & - & - & - & - & - \\
		F & - & - & - & - & - & - \\
	\end{tabular}
\end{minipage}%
\end{figure}


\begin{figure}[h]
\centering
\begin{minipage}{.5\textwidth}
	\centering
	\begin{tabular}{c c l}
		$t = 1$ & : & $\delta(B,\, F) = 22$ \\
		$t = 2$ & : & $\delta(A,\, C) = 22$ \\
		& & $\delta(C,\, F) = 26$
	\end{tabular}
\end{minipage}%
\begin{minipage}{.5\textwidth}
	\centering
	\begin{tabular}{c c l}
		$t = 3$ & : &$\delta(A,\, D) = \delta(A,\, C) + \delta(C,\, D) = 22 + 2 = 24$ \\
		& & $\delta(D,\, F) = \delta(D,\, C) + \delta(C,\, F) = 2 + 24 = 26$ \\
		& & $\delta(B,\, D) = \delta(B,\, C) + \delta(C,\, D) = 2 + 2 = 4$
	\end{tabular}
\end{minipage}%
\end{figure}

\subsection*{Implémentation}
\lstinputlisting{Scripts/Floyd_Warshall.py}