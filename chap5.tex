\chapter{Parcours d'arbres k-aire}

%WARNING
On appelle arbre $k$-aire un arbre dont chaque nœud a au plus $k$ fils.

\section*{Problème}

Étant donné un arbre/graphe, on veut parcourir chaque sommet une et une seule fois.

\section{Structure de donnée utilisée}

Une file $Q$, \textit{FIFO (First In First Out)}, avec les opérations suivantes :

\begin{itemize}
\item \textcolor{ocre}{$Q.push(v)$} : ajouter $v$ à la fin de la file.
\item \textcolor{ocre}{$u = Q.pop()$} : retire de le premier élément de la liste et le place dans la variable $u$.
\item \textcolor{ocre}{$Q.vide()$} : retourne Vrai si la file est Vide, Faux sinon.
\end{itemize}

On ajoute les informations suivantes aux sommets au moment du parcours, \textbf{pour chaque sommet $A$} :

\begin{itemize}
\item \textcolor{ocre}{Datage} : $A.debut$, $A.fin$ les instants où on découvre/quitte $A$
\item \textcolor{ocre}{Couleur} :
	\begin{itemize}
		\item blanc : pas encore visité
		\item gris : en cours
		\item noir : terminé
	\end{itemize}
\item \textcolor{ocre}{Distance} : pour le parcours en largeur, la distance à la racine
\end{itemize}

\lstinputlisting{Scripts/Noeud.py}

On traite les arbres à arité (nombre d'enfants) variable. Chaque sommet $A$ peut avoir un nombre arbitraire de fils donnés dans une liste $A.fils$.

\section{Parcours en largeur des arbres} \index{Parcours en largeur}

\lstinputlisting{Scripts/BFS.py}

\section{Propriétés du parcours en largeur}

\begin{proposition}\label{prop:p1}
Les sommets sont visités par ordre de profondeur.
\end{proposition}
On découpe en deux étapes la preuve de la proposition \ref{prop:p1}.

\begin{proposition}\label{prop:p1a}
Supposons que la file contienne les valeurs $(v_{1},\, d_{1}),\, ...\, ,\, (v_{k},\, d_{k})$, lorsque $(v_{i},\, d_{i})$ entre dans la file $d_{i} = Prof_{A}(v_{i})$.
\end{proposition}

\textbf{\sffamily\small Preuve de la proposition \ref{prop:p1a}}\\
Par induction sur $t$.
\begin{itemize}
\item Au temps $t = 0$ : La file contient $(A,\, 0)$ et par définition $Prof_{A}(t) = 0$.

\item Au temps $t + 1$ : Supposons que $Q$ contienne $(v_{1},\, d_{1}),\, ...\, ,\,  (v_{k},\, d_{k})$ où $d_{i} = Prof_{A}(v_{i})$. On retire $(v_{1},\, d_{1})$ de la file. On ajoute $(f_{1},\, d_{1} + 1),\, ...\, ,\, (f_{e},\, d_{1} + 1)$ où les $f_{i}$ sont les fils de $v_{1}$. Les $f_{i}$ sont bien à hauteur $d_{1} + 1$ dans l'arbre.
\end{itemize}

\begin{proposition}\label{prop:p1b}
Si au temps $t$, $Q$ contient $(v_{1},\, d_{1}),\, ...\, ,\, (v_{k},\, d_{k})$ alors $d_{1} \leq d_{2} \leq \, ... \, \leq d_{k}$ et $d_{k} - d_{1} \leq 1$.
\end{proposition}

\textbf{\sffamily\small Preuve de proposition \ref{prop:p1b}}\\
Par induction sur $t$.
\begin{itemize}
\item Au temps $t = 0$ : La file contient $(A,\, 0)$, la proposition est donc vraie.
\item Supposons qu'à l'instant $t$, $Q$ contienne $(v_{1},\, d_{1}),\, ...\, ,\,  (v_{k},\, d_{k})$, par hypothèse d'induction, ceux-ci vérifient la proposition. On retire $(v_{1},\, d_{1})$ et on ajoute ses fils $(f_{1},\, d_{1} + 1),\, ...\, ,\, (f_{e},\, d_{1} + 1)$ de $v_{1}$ à la suite.
\end{itemize}

\vspace{1cm}
Vérification : 2 cas possibles $d_{1} = d_{k}$ ou $d_{1} + 1 = d_{k}$ (par hypothèse d'induction). On a  $(v_{2},\, d_{2}),\, ...\, ,\,  (v_{k},\, d_{k})$ et $(f_{1},\, d_{1} + 1),\, ...\, ,\, (f_{e},\, d_{1} + 1)$.

\begin{enumerate}

\item Les distances sont encore triées, elles sont triées jusqu'à $d_{k}$, puis pour $f_{1},\, ...\,,\, f_{e}$ : 

\begin{minipage}[t]{\textwidth}
	\begin{itemize}
		\item $d_{k} = d_{1}$ , on ajoute $d_{1},\, ...\,,\, d_{1}$.
		\item $d_{k} = d_{1} + 1$ , on ajoute $d_{1} + 1,\, ... \,,\, d_{1} + 1$. 
	\end{itemize}
\end{minipage}

\item $d_{1} + 1 - d_{2} < 1$ ? Par hypothèse, $d_{1} \leq d_{2} \leq d_{1} + 1$, ok.

\end{enumerate}

\begin{proposition}\label{prop:p2}
Chaque sommet entre et sort de la file une et une seule fois.
\end{proposition}

\textbf{\sffamily\small Preuve de la proposition \ref{prop:p2}}\\
\begin{enumerate}
\item Chaque sommet qui entre dans la file doit en sortir : l'algorithme ne se termine que lorsque la file est vide.

\item Un sommet qui est entré dans la file ne peut pas y ré-entrer. En effet, il ne rentre que quand son père sort. Mais par récurrence son père ne peut pas ré-entrer dans la file.

\item Chaque sommet de l'arbre entre dans la file. Sinon, son père n'est pas entré, ainsi de suite tous ses ascendants ne sont pas rentrés jusqu'à la racine.
\end{enumerate}

\textcolor{ocre}{Conclusion} : Le parcours en largeur parcourt les sommets une et une seule fois par ordre croissant de profondeur. La complexité est $2 \vert A \vert$ (en nombre de $pop()$ et $push(v)$).

\section{Parcours en profondeur des arbres}

Parcours en profondeur des arbres à arité variable :

\lstinputlisting{Scripts/DFS.py}

\subsection*{Exemple}

\begin{figure}[h]
	\centering
	\begin{tikzpicture}
		\GraphInit[vstyle=Normal]
		\tikzset{VertexStyle/.append  style={fill = node}}
		\tikzset{VertexStyle/.append  style={text = white}}
		\SetGraphUnit{1.5}
		\Vertex{4}
		\SOWE(4){3}
		\Edge(4)(3)
		\SOEA(4){6}
		\Edge(4)(6)
		\SOWE(3){1}
		\Edge(3)(1)
		\SOWE(1){2}
		\Edge(1)(2)
		\SOWE(6){5}
		\Edge(6)(5)
		\SOEA(6){7}
		\Edge(6)(7)
		\SetUpEdge[color=cblue,lw=1.5]
		\Edge[style={->, bend right}](4)(3)
		\Edge[style={->, bend right}](3)(1)
		\Edge[style={->, bend right}](1)(2)
		\SetUpEdge[color=orange,lw=1.5]
		\Edge[style={->, bend right}](2)(1)
		\Edge[style={->, bend right}](1)(3)
		\Edge[style={->, bend right}](3)(4)
		\SetUpEdge[color=cblue,lw=1.5]
		\Edge[style={->, bend right}](4)(6)
		\Edge[style={->, bend right}](6)(5)
		\SetUpEdge[color=orange,lw=1.5]
		\Edge[style={->, bend right}](5)(6)
		\SetUpEdge[color=cblue,lw=1.5]
		\Edge[style={->, bend right}](6)(7)
		\SetUpEdge[color=orange,lw=1.5]
		\Edge[style={->, bend right}](7)(6)
		\Edge[style={->, bend right}](6)(4)
	\end{tikzpicture}
	\label{fig:graph2}
\end{figure}

\textcolor{ocre}{Remarque} : si on associe au parcours en profondeur $\text{"je vais visiter i" } \mapsto \textcolor{cblue}{\underset{i}{(}}$ et $\text{"je quitte i" }  \mapsto \textcolor{orange}{\underset{i}{)}}$ on obtient une expression correctement parenthésée ; ici : $\textcolor{cblue}{\underset{4}{(}} \; \textcolor{cblue}{\underset{3}{(}} \; \textcolor{cblue}{\underset{1}{(}} \; \textcolor{cblue}{\underset{2}{(}} \; \textcolor{orange}{\underset{2}{)}} \; \textcolor{orange}{\underset{1}{)}} \; \textcolor{orange}{\underset{3}{)}} \; \textcolor{cblue}{\underset{6}{(}} \; \textcolor{cblue}{\underset{5}{(}} \; \textcolor{orange}{\underset{5}{)}} \; \textcolor{cblue}{\underset{7}{(}} \; \textcolor{orange}{\underset{7}{)}} \; \textcolor{orange}{\underset{6}{)}} \; \textcolor{orange}{\underset{4}{)}}$

\section{Propriétés du parcours en profondeur des arbres}

\begin{proposition}\label{prop:p3}
Le parcours en profondeur a les propriétés suivantes :
\begin{itemize}
\item $\forall A$ $A.debut < A.fin$
\item Avant $A.debut$ , $\;A.couleur = \text{"blanc"}$
\item Entre $A.debut$ et $A.fin$ , $\;A.couleur = \text{"gris"}$
\item Après $A.fin$ , $\;A.couleur = \text{"noir"}$
\end{itemize}

\end{proposition}

\begin{proposition}\label{prop:p4}
Pour toute paire de sommets $A,\, A^{\prime}$ les intervalles $I_{A} = (A.debut,\, A.fin)$ et $I_{A^{\prime}} = (A^{\prime}.debut,\, A^{\prime}.fin)$ ne peuvent se chevaucher.
\end{proposition}

\textbf{\sffamily\small Preuve de la proposition \ref{prop:p4}}\\
Supposons $A.debut < A^{\prime}.debut < A.fin$ , alors entre $A.debut$ et $A.fin$, les appels récursifs, y compris à $DFS(A)$ se sont terminés donc $A^{\prime}.debut < A^{\prime}.fin < A.fin$.

\begin{proposition}
$A^{\prime}$ est descendant de $A$ $\Leftrightarrow$ $A.debut < A^{\prime}.debut < A^{\prime}.fin < A.fin$
\end{proposition}

\begin{proposition}\label{prop:p5}
L'ensemble des sommets gris forment un chemin de la racine jusqu'à un sommet de l'arbre.
\end{proposition}

\textbf{\sffamily\small Preuve de la proposition \ref{prop:p5}}
Soient $v_{1},\, ...\, ,\, v_{k}$ les sommets gris à l'instant $t$. Supposons $v_{1}.debut < v_{2}.debut < ...\, < v_{1}.fin$

On a montré : $v_{1}.debut < v_{2}.debut < ...\, < v_{1}.fin$

D'après la proposition \ref{prop:p4}, on a que $v_{i+1}$ descendant de $v_{i}$ $\forall i$ $1 \leq i \leq k$. Mais il faut monter que $v_{i+1}$ est fils de $v_{i}$. [...]


%----------------------------------------------------------------------------------------
%	GRAPHES
%----------------------------------------------------------------------------------------

\chapter{Parcours de graphes}

\section{Représentation des graphes (non orientés)}

On ajoute les informations suivantes \textbf{pour chaque sommet $S$} :

\begin{itemize}
\item \textcolor{ocre}{$S.debut$, $S.fin$} : pour le datage, les instants où on découvre/quitte $S$
\item \textcolor{ocre}{$S.couleur$} :
	\begin{itemize}
		\item blanc : pas encore visité
		\item gris : en cours
		\item noir : terminé
	\end{itemize}
\item \textcolor{ocre}{$S.dist$} : pour le parcours en largeur, la distance à la racine
\item \textcolor{ocre}{$S.voisins$} : la liste des voisins.
\item \textcolor{ocre}{$S.pred$} : le prédécesseur de $S$ dans le parcours.
\end{itemize}

\lstinputlisting{Scripts/Noeud_Graph.py}

\section{Parcours en largeur des graphes}\index{Parcours en largeur}

\textcolor{ocre}{Problème 1} : si on applique l'algorithme des arbres, on obtient une boucle infinie.

\textcolor{ocre}{Problème 2} : certains sommets ne seront jamais visités.

Supposons à partir de maintenant que $G$ est connexe.

\begin{definition}[Chemins]\index{Chemins} Notations : \\
$u \rightarrow^{d}_{G} v$ : il existe un chemin de longueur $d$ de $u$ à $v$ dans $G$.\\
$u \rightarrow^{\ast}_{G} v$ : il existe un chemin de $u$ à $v$ dans $G$.\\
$\delta_{G}(u,\, v)$ : $\min \lbrace d : u \rightarrow^{d}_{G} v \rbrace$\\
$V_{k}(s)$ : $\lbrace v \in S : \delta_{G}(u,\, v) = k\rbrace$
\end{definition}

\lstinputlisting{Scripts/BFS_Graph.py}

\begin{proposition}\label{prop:p6}
À l'appel de $BFS(s)$ :

\begin{enumerate}
\item Les sommets (accessibles depuis $s$) sont parcourus par ordre de distance à $s$.
\item Après le parcours d'un sommet $u$, $u.dist$ est égal à la distance de $u$ à $s$.
\item Si le sommet $u$ est à distance $d$ de $s$, alors il existe un chemin de $s$ à $u$ de longueur $d$ dont la dernière arête est $(u.pred, u)$.
\end{enumerate}
\end{proposition}


\textbf{\sffamily\small Preuve de la proposition \ref{prop:p6}.1}

Il faut se convaincre qu'à l'instant où on traite le dernier sommet à distance $d$ de $s$, tous les sommets à distance $d + 1$ soient dans la file.

On a montré pour les arbres que la file ne contient que des sommets à distance $d + 1$ ; il faut montrer que tous sont dans la file.

Posons $u \in V_{d+1}$ quelconque.

Comme $\delta(s,\, u) = d + 1$ , $\exists v$ , $\delta(s,\, u) = d$ et $(u,\, v) \in A$.

Par induction, $v$ a été traité avec les sommets de distance $d$ et donc $u$ :

\begin{itemize}
\item s'il n'avait pas déjà été visité au moment de traiter $v$, a été ajouté à la file 
\item s'il avait déjà été visité, $u$ a déjà été mis dans la file. Il a dû être mis dans la  file pour un autre sommet $v^{\prime}$.
\end{itemize}
$\delta(s,\, v^{\prime})$ est nécessairement égal à $d$ sinon $\delta(s,\, u) = \delta(s,\, v^{\prime}) + 1 < d + 1$.

\vspace{0.2cm}
\begin{proposition}\label{prop:p7}
Tous les sommets accessibles à partir de $s$ sont visités une et une seule fois.
\end{proposition}

\begin{definition}[Graphe des prédécesseurs]\index{Graphe des prédécesseurs}
Après le parcours ($BFS$ ou $DFS$) à partir de $s \in S$ d'un graphe $G = (S,\, A)$ , on appelle graphe des prédécesseurs :
$G_{pred} = (S,\, A^{\prime})$ avec $A^{\prime} = \lbrace(u,\, u.pred),\; u \in S \setminus \lbrace S \rbrace \rbrace$
\end{definition}

\begin{proposition}\label{prop:arbre de parcours}\index{Arbre de parcours}
Si $G$ est connexe, $G_{pred}$ est un arbre ; on l'appelle arbre de parcours.
\end{proposition}

\textbf{\sffamily\small Preuve de la proposition \ref{prop:arbre de parcours}}

$G_{pred}$ a $n = \vert S \vert$ sommets et $n - 1$ arêtes ; $G_{pred}$ est connexe car on a parcouru tous les sommets en passant par des arêtes du graphe.

\begin{proposition}
La complexité de $BFS(S)$ en nombre de comparaisons est : $\sum\limits_{x \in S} \sum\limits_{v \in x} 1 = 2 \vert A \vert$ (car $\sum\limits_{x \in S}$ représente la boucle \texttt{while}, 1 fois pour chaque sommet retiré de la file ; $\sum\limits_{v \in x}$ la boucle \texttt{for})
\end{proposition}

\newpage

\section{Parcours en profondeur des graphes}\index{Parcours en profondeur}

\subsection{Version récursive}

\lstinputlisting{Scripts/DFS_Graph.py}

\subsection{Version itérative}

\lstinputlisting{Scripts/DFS_Graph_iter.py}

\subsection*{Exemple}

\begin{figure}[h]
	\centering
	\begin{tikzpicture}
		\GraphInit[vstyle=Normal]
		\tikzset{VertexStyle/.append  style={fill = node}}
		\tikzset{VertexStyle/.append  style={text = white}}
		\SetGraphUnit{0.75}
		\Vertex[x=2 , y=0]{1}
		\Vertex[x=0 , y=-2]{2}
		\Vertex[x=2 , y=-2]{4}
		\Vertex[x=4 , y=-2]{5}
		\Vertex[x=1 , y=-4]{3}
		\Vertex[x=3 , y=-4]{6}
		\Vertex[x=5 , y=-4]{7}
		\Edge(1)(2)
		\Edge(1)(4)
		\Edge(1)(5)
		\Edge(2)(3)
		\Edge(4)(3)
		\Edge(4)(6)
		\Edge(2)(4)
		\Edge(4)(5)
		\Edge(3)(6)
		\Edge(6)(7)
		\SetUpEdge[color=cblue,lw=1.5]
		\Edge[style={->, bend right=20}](1)(2)
		\Edge[style={->, bend right=20}](2)(3)
		\Edge[style={->, bend right=20}](3)(4)
		\Edge[style={->, bend right=20}](4)(6)
		\Edge[style={->, bend right=20}](6)(7)
		\Edge[style={->, bend right=20}](4)(5)
		\SetUpEdge[color=orange,lw=1.5]
		\Edge[style={->, bend right=20}](7)(6)
		\Edge[style={->, bend right=20}](6)(4)
		\Edge[style={->, bend right=20}](5)(4)
		\Edge[style={->, bend right=20}](4)(3)
		\Edge[style={->, bend right=20}](3)(2)
		\Edge[style={->, bend right=20}](2)(1)
	\end{tikzpicture}
	\caption{$G$}
\end{figure}

\textcolor{ocre}{Remarque} : si on associe au parcours en profondeur $\text{"je vais visiter i" } \mapsto \textcolor{cblue}{\underset{i}{(}}$ et $\text{"je quitte i" }  \mapsto \textcolor{orange}{\underset{i}{)}}$ on obtient une expression correctement parenthésée ; ici : $\textcolor{cblue}{\underset{1}{(}} \; \textcolor{cblue}{\underset{2}{(}} \; \textcolor{cblue}{\underset{3}{(}} \; \textcolor{cblue}{\underset{4}{(}} \; \textcolor{cblue}{\underset{6}{(}} \; \textcolor{cblue}{\underset{7}{(}} \; \textcolor{orange}{\underset{7}{)}} \; \textcolor{orange}{\underset{6}{)}} \; \textcolor{cblue}{\underset{5}{(}} \; \textcolor{orange}{\underset{5}{)}} \; \textcolor{orange}{\underset{4}{)}} \; \textcolor{cblue}{\underset{3}{(}} \; \textcolor{orange}{\underset{2}{)}} \; \textcolor{orange}{\underset{1}{)}}$

\begin{definition}[Chemins]\index{Chemins} On écrit :

\begin{itemize}
\item $u \rightarrow_{G}^{\ast} v$ : il existe un chemin de $u$ à $v$ dans $G$.
\item $u \Rightarrow_{G}^{\ast} v$ : il existe un chemin orienté de $u$ à $v$ dans $G_{pred} = \lbrace (s.pred,\, s),\; s\in S\rbrace$.
\end{itemize}
\end{definition}

\begin{theorem}[Chemin blanc]
Si $u \Rightarrow_{G_{pred}}^{\ast} v$ $\Leftrightarrow$ au temps $t = u.debut$ , un chemin blanc relie $u$ à $v$ dans $G$.
\end{theorem}

\newpage

\begin{figure}[h]
	\centering
	\begin{tikzpicture}
		\GraphInit[vstyle=Normal]
		\tikzset{VertexStyle/.append  style={fill = node}}
		\tikzset{VertexStyle/.append  style={text = white}}
		\SetGraphUnit{0.75}
		\Vertex[x=2 , y=0]{1}
		\Vertex[x=0 , y=-2]{2}
		\Vertex[x=2 , y=-2]{4}
		\Vertex[x=4 , y=-2]{5}
		\Vertex[x=1 , y=-4]{3}
		\Vertex[x=3 , y=-4]{6}
		\Vertex[x=5 , y=-4]{7}
		\Edge[style={->}](1)(2)
		\Edge[style={->}](2)(3)
		\Edge[style={->}](3)(4)
		\Edge[style={->}](4)(6)
		\Edge[style={->}](4)(5)
		\Edge[style={->}](6)(7)
	\end{tikzpicture}
	\caption{$G_{pred}$}
\end{figure}

\begin{proposition}\label{prop:dfs_graph_p1}
Le parcours en profondeur a les propriétés suivantes : $\forall$ sommet $u \in S$

\begin{enumerate}
\item $u.debut < u.fin$
\item $\forall t < u.debut$ $\; u.couleur = \text{blanc}$
\item $\forall t$ : $\, u.debut \leq t \leq u.fin$ $\;u.couleur = \text{gris}$
\item $\forall t > u.fin$ $\; u.couleur = \text{noir}$
\end{enumerate}
\end{proposition}


\begin{proposition}\label{prop:dfs_graph_p2}
Pour toute paire de sommets $u,\, v$ les intervalles $I_{u} = (u.debut,\, u.fin)$ et $I_{v} = (v.debut,\, v.fin)$ ne peuvent se chevaucher.
\end{proposition}

\begin{proposition}\label{prop:dfs_graph_p3}
Pour toute paire de sommets $(u\, v)$ $\;u \Rightarrow_{G}^{\ast} v$ $\Leftrightarrow$ $u.debut < v.debut < v.fin < u.fin$
\end{proposition}

\textbf{\sffamily\small Preuve du théorème du "chemin blanc"}

\begin{itemize}
\item[$(\Rightarrow)$] Supposons que $u \Rightarrow_{G_{pred}}^{\ast} v$ par le chemin $u = u_{0} \Rightarrow_{G_{pred}}^{1} u_{1} \Rightarrow_{G_{pred}}^{1} \,...\; \Rightarrow_{G_{pred}}^{1} u_{k} = v$ .

D'après la proposition \ref{prop:dfs_graph_p3} les intervalles s'emboîtent : $\underset{u_{0}}{(} \; \underset{u_{1}}{(} \; \underset{u_{2}}{(} \; ... \; \underset{u_{k}}{(} \; \underset{u_{k}}{)} \; ... \; \underset{u_{2}}{)} \; \underset{u_{1}}{)} \; \underset{u_{0}}{)}$

D'après la proposition \ref{prop:dfs_graph_p1} : au temps $t = u.debut$ $\;u_{1},\, u_{2},\; ...\; u_{k}$ sont blancs.

Donc le chemin $u = u_{0} \Rightarrow_{G_{pred}}^{1} u_{1} \Rightarrow_{G_{pred}}^{1} \,...\; \Rightarrow_{G_{pred}}^{1} u_{k} = v$ est blanc.

\item[$(\Leftarrow)$] Supposons qu'il existe un chemin blanc de $u$ à $v$ dans $G$ au temps $t = u.debut$

$u = u_{0} \Rightarrow_{G_{pred}}^{1} u_{1} \Rightarrow_{G_{pred}}^{1} \,...\; \Rightarrow_{G_{pred}}^{1} u_{k} \Rightarrow \, ...\; \Rightarrow u_{i-1} \rightarrow u_{i} \rightarrow \,...\; u_{k} = v$

Soit $i$ le point le plus petit tel que $(u_{i - 1},\, u_{i}) \notin G_{pred}$ .

%WARNING
On va montrer qu'on peut trouver un chemin dans $G_{pred}$.

Posons $k < i$ le plus petit tel que $(u_{k},\, u_{i}) \in G$ .

Examinons la structure au moment $t^{\prime} = u_{k}.fin$ .

On veut montrer qu'on peut aller jusqu'à $i$ avec un chemin dans $G_{pred}$ .

Au temps $t^{\prime}$ , quelle est la couleur de $u_{i}$ ?

\begin{itemize}[label=$\bullet$]
\item $u_{i} = \text{blanc}$ ? Non, on visite tous les voisins blancs de $u_{k}$ avant $u_{k}.fin$
\item $u_{i} = \text{gris}$ ? On avait $u_{0} \Rightarrow^{\ast} u_{i} \Rightarrow u_{k}$ or $k < i$ , donc impossible
\item $u_{i} = \text{noir}$ ? 2 possibilités :
\begin{itemize}[label=$\circ$]
\item Soit $\underset{u_{0}}{(} \;\; \underset{u_{i}}{(} \; \; \underset{u_{i}}{)} \; \underset{u_{k}}{(}\; \underset{u_{k}}{)} \; \underset{u_{0}}{)}$ d'après la proposition \ref{prop:dfs_graph_p2}
\item Soit $\underset{u_{0}}{(} \;\; \underset{u_{k}}{(} \; \underset{u_{i}}{(} \; \; \underset{u_{i}}{)} \; \underset{u_{k}}{)} \; \underset{u_{0}}{)}$ d'après la proposition \ref{prop:dfs_graph_p3}.
\end{itemize}
Dans les deux cas $u_{0} \Rightarrow_{G_{pred}}^{\ast} u_{i}$
\end{itemize}
\end{itemize}

\subsubsection*{Applications des parcours de graphes}
\begin{itemize}
\item Recherche de cycles (deadlock, BDD)
\item Plus courts chemins dans les graphes
\item Détection de composante connexe (reconnaissance de motifs dans une image)
\item Tri topologique (gestion de projet)
\item Recherche de stratégie optimale dans les jeux.
\end{itemize}