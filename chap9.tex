\chapter{Les plus cours chemins : Dijkstra}\index{Dijkstra}

\section*{Communisme vs Capitalisme}
\begin{figure}[h]
\centering
\begin{minipage}{.5\textwidth}
  \centering
  	\begin{tikzpicture}
		\GraphInit[vstyle=Normal]
		\tikzset{VertexStyle/.append  style={fill = node}}
		\tikzset{VertexStyle/.append  style={text = white}}
		\SetGraphUnit{1.5}
		\Vertex{A}
		\SOWE(A){B}
		\SOEA(A){C}
		\Edge[label=$5$](A)(C)
		\SetUpEdge[color=orange,lw=1.5]
		\Edge[label=$5$](A)(B)
		\Edge[label=$3$](B)(C)
	\end{tikzpicture}
  \caption{Arbre Couvrant minimal, coût = $8$}
\end{minipage}%
\begin{minipage}{.5\textwidth}
  \centering
  	\begin{tikzpicture}
		\GraphInit[vstyle=Normal]
		\tikzset{VertexStyle/.append  style={fill = node}}
		\tikzset{VertexStyle/.append  style={text = white}}
		\SetGraphUnit{1.5}
		\Vertex{A}
		\SOWE(A){B}
		\SOEA(A){C}
		\Edge[label=$3$](B)(C)
		\SetUpEdge[color=orange,lw=1.5]
		\Edge[label=$5$](A)(B)
		\Edge[label=$5$](A)(C)
	\end{tikzpicture}
  \caption{Plus courts chemins depuis A}
\end{minipage}%
\end{figure}

Étant donné un graphe valué $G,\, w$ et un sommet de départ $s$. On veut calculer les plus courts chemins de $s$ à $v$ dans $G$ $\forall v \in S$.

\begin{definition}[Notations]
Si $C = (s = u_{0},\, u_{1},\, u_{2},\, ...\; u_{k} = v)$ est un chemin de $s$ à $v$ dans $G$, on écrit :

\begin{itemize}
\item $s \overset{C}{\longrightarrow}_{G} v$
\item $w(C) = \sum\limits_{i = 1}^{k} w (u_{i-1},\, u_{i})$
\item $\delta (s,\, v) = min \lbrace w (C) : s \overset{C}{\longrightarrow}_{G} v\rbrace$
\end{itemize}
\end{definition}

\begin{proposition}
On peut toujours trouver des chemins $C_{u} s \overset{C}{\longrightarrow} u$ tel que l'union des $C_{u}$ forme un arbre.
\end{proposition}

\textbf{\sffamily\small Preuve}
Supposons qu'on ait 2 sommets $u$, $v$ des plus courts chemins $C_{u}$, $C_{v}$ dont l'union forme un cycle.

%WARNING
Les segments $C_{1}$, $C_{2}$ $\ x \overset{C_{1}}{\longrightarrow} y$ $\ x \overset{C_{2}}{\longrightarrow} y$ ont le même coût, sinon supposons $w(C_{1}) < w(C_{2})$ alors il y a un chemin plus court de $s$ à $v$ qui passe par $C_{1}$; $C_{2}$ n'est pas un plus court chemin, impossible.

\section{Algorithme de Dijkstra}

On modifie l'algorithme de Prim, la notion de priorité change, maintenant c'est la distance à $s$.

améliorer la priorité $\equiv$ améliorer la distance à $s$

\lstinputlisting{Scripts/Dijkstra.py}

\section{Preuve de l'algorithme de Dijkstra}

\begin{definition}
Notons $u.prio^{t}$ la valeur de $u.prio$ à la $t^{\text{ième}}$ itération.
\end{definition}

\begin{theorem}
À l'itération $t$, si $u \notin$ $tas$ alors $u.prio^{t} = \delta(s,\, u)$.
\end{theorem}

On utilise sans démontrer trois propriétés :

\begin{enumerate}[label=(P\arabic*)]
\item S'il n'y a pas de chemin de $s$ à $u$ alors $u.prio^{t} = \infty \ \forall t$
\item "Borne supérieure" : $\forall u,\; \forall \ u.prio^{t} \geq \delta(s,\, u)$ et $u.prio^{t} \geq u.prio^{t+1}$
\item Si $s \longrightarrow_{G}^{\ast} u \longrightarrow^{1} v$ est un plus court chemin de $s$ à $v$ et $u.prio^{t} = \delta(s,\, u)$,\\ alors $\forall t^{\prime} > t \ v.prio^{t^{\prime}} = \delta(s,\, v)$
\end{enumerate}

\vspace*{0.16cm}
\textbf{\sffamily\small Preuve du théorème}

Supposons qu'à l'itération $t$, $u$ sorte du $tas$.

Regardons $T^{t}$ l'ensemble des arêtes déjà choisies.

Soit $C$ un plus court chemin de $s$ à $u$, et $x,\, y$ la première arête de $C$ telle que $(x,\, y) \notin T^{t}$.

$s\overset{C_{1}\subseteq T}{\longrightarrow} x \overset{\notin T}{\longrightarrow}^{1} y\overset{C_{2}}{\longrightarrow} u$

\textbf{On va montrer que $y = u$}.

\begin{enumerate}[label=(\Roman*)]
\item Montrons que : $y.prio^{t} = \delta(s,\,y)$

\textbf{(*)} On utilise le fait que sur un plus court chemin de $u_{0}$ à $u_{k}$ les segments initiaux de $u_{0}$ à $u_{i}$ (avec $0 \leq i \leq k$) sont aussi des plus courts chemins

On a que $x \in T^{t}$, supposons qu'il a été ajouté à l'instant $t^{\prime} < t$.

Par hypothèse d'induction : $x.prio^{t^{\prime}} = \delta(s,\, x)$

À l'itération $t^{\prime}$, $y$ est encore dans le tas et $y$ est voisin de $x$, et on aura posé $y.prio^{t^{\prime}} \leq x.prio^{t^{\prime}} + w(x,\, y)$ or $s \longrightarrow^{i} x \longrightarrow^{1} y$ est un plus court chemin (par \textbf{(*)}).

$y.prio^{t^{\prime}} \leq \delta(s,\, x) + w(x,\, y) = $ longueur du chemin $ = \delta(s,\, y)$ et par P2 $y.prio^{t} = \delta(s,\, y)$

\item On montre que $C_{2} = \varnothing$

Donc on conclut que $y = u$, et par (I) $u.prio^{t} = y.prio^{t} = \delta(s,\, y) = \delta(s,\, u)$

À l'itération $t$, $y \in T^{t}$ , mais y a une priorité $\neq \infty$ donc $y$ est dans le tas, tout comme $u$ qui a été choisi.

\begin{tabular}{l l l}
	$u.prio^{t}$ & $\leq$ & $y.prio^{t}\quad$ car $u$ sort du tas\\
	& $=$ & $\delta(s,\, y)\quad$ (II)\\
	& $\leq$ & $\delta(s,\, u)\quad$ car $y$ précède $u$ sur un plus court chemin\\
	& $\leq$ & $u.prio^{t}\quad$ (P2)
\end{tabular}

$\Rightarrow$ toutes ces quantités sont égales $\Rightarrow$ $y = u$

\end{enumerate}
