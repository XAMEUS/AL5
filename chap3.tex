\chapter{L'algorithme : Union-Find}

\section{Définition}

On démarre avec des éléments d'un univers $U$ (ici les sommets d'un graphe). On admet 2 opérations :
\begin{itemize}
\item\index{UNION} $UNION (u, v)$ : associe $u$ et $v$ (ici, on ajoute une arête)
\item\index{FIND} $FIND (u, v)$ : permet de savoir si $u$ et $v$ sont associés, transitive (ici, si il existe un chemin de $u$ à $v$)
\end{itemize}

\section{Exemple}

\begin{algorithm}[h]
\caption{Algorithme de Kruskal}
\begin{algorithmic}
\FOR{$(u, v)$ in $A$ (par ordre croisant d'étiquette)}
\IF{\NOT $FIND (u, v)$} \STATE{$UNION (u, v)$} \ELSE \STATE{end} \ENDIF
\ENDFOR
\end{algorithmic}
\end{algorithm}

\section{Applications}
\begin{itemize}
\item Segmentation d'image (combien de formes dans une image).
\item Construction de labyrinthes.
\item Kruskal, arbre couvrant minimal.
\item Détection de cycles.
\end{itemize}

\section{Idée de l'algorithme}

Chaque élément connait son "chef".

On note $Id[u]$ le chef de $u$.

Pour savoir si $u$ et $v$ sont associés, on note la "hiérarchie" de chef en chef. Si $u$ et $v$ ont le même "PDG", alors $u$ et $v$ sont associés.

Pour associer $u$ et $v$ : $UNION(u, v)$ = le chef du "PDG" de $u$ devient le "PDG" de $v$.

\begin{figure}[h]
	\centering
	\begin{tikzpicture}
		\GraphInit[vstyle=Normal]
		\tikzset{VertexStyle/.append  style={fill = node}}
		\tikzset{VertexStyle/.append  style={text = white}}
		\SetGraphUnit{1.5}
		\SetVertexNormal[LineWidth=2pt]
		\SetVertexNormal[FillColor=ocre]
		\Vertex{v}
		\SetVertexNormal[LineColor=orange,LineWidth=2pt, FillColor=ocre]
		\SetUpEdge[color=black,lw=1.5]
		\SetVertexLabel \NO(v){w} \Edge[style={->}](v)(w)
		\SetVertexNormal[LineColor=orange,LineWidth=2pt, FillColor=ocre]
		\SetUpEdge[color=black,lw=1.5]
		\SetVertexLabel \SOEA(v){u} \Edge[style={->}](u)(v)
		\SetVertexNormal[FillColor=ocre]
		\SetUpEdge[color=black]
		\SetVertexNoLabel \SO(v){z} \Edge[style={->}](z)(v)
		\SetVertexNoLabel \SOWE(w){x} \Edge[style={->}](x)(w)
		\SetVertexNoLabel \SO(x){y} \Edge[style={->}](y)(x)
		\SetVertexNoLabel \SOEA(y){yx} \Edge[style={->}](yx)(y)
		\SetVertexNoLabel \SOWE(y){xx} \Edge[style={->}](xx)(y)
		\SetVertexNoLabel \SOWE(x){yy} \Edge[style={->}](yy)(x)
		\SetGraphUnit{8}
		\SetVertexNormal[LineColor=orange,LineWidth=2pt, FillColor=ocre]
		\SetVertexLabel \EA(w){E}
		\SetUpEdge[color=cblue,lw=1.5]
		\Edge[style={->, bend left}](w)(E)
		\SetGraphUnit{1.5}
		\SetVertexNormal[FillColor=ocre]
		\SetUpEdge[color=black,lw=1.5]
		\SetVertexLabel \SOEA(E){D} \Edge[style={->}](D)(E)
		\SetUpEdge[color=black,lw=1.5]
		\SetVertexLabel \SOWE(D){C} \Edge[style={->}](C)(D)
		\SetVertexNormal[LineColor=orange,LineWidth=2pt, FillColor=ocre]
		\SetUpEdge[color=black,lw=1.5]
		\SetVertexLabel \SOWE(C){B} \Edge[style={->}](B)(C)
		\SetVertexNormal[FillColor=ocre]
		\SetUpEdge[color=black]
		\SetVertexLabel \SOWE(B){A} \Edge[style={->}](A)(B)
		\SetVertexNoLabel \SOWE(E){F} \Edge[style={->}](F)(E)
		\SetVertexNoLabel \SOWE(F){G} \Edge[style={->}](G)(F)
		\SetVertexNoLabel \SOWE(F){H} \Edge[style={->}](H)(F)
		
	\end{tikzpicture}
	\caption{$UNION(u, B)$}
\end{figure}

\section{Retour sur Kruskal}
On teste successivement les arêtes avec $FIND$ , tant que la fonction renvoie $FAUX$ on utilise $UNION$, sinon on a fini.

\section{Implémentation en Python}

La classe UF en Python :

\lstinputlisting{Scripts/UF.py}

\section{Complexité}

\begin{itemize}
	\item \texttt{root1} : $O(\text{hauteur de l'arbre})$
	\item \texttt{find1} : $2 \times \text{hauteur} + 1$
	\item \texttt{union1} : $4 \times \text{hauteur} + 1$
\end{itemize}


\section{$UNION$ : Maîtriser la hauteur des arbres}
Pour améliorer la complexité on cherche à maîtriser la hauteur des arbres.

Suggestion : on raccroche le plus petit arbre au plus grand ; cela n'augmente pas la hauteur du nouvel arbre sauf si les deux sont de même hauteur.

On maintient un deuxième tableau, qui, pour chaque élément donne sa hauteur.

Au moment de faire l'union, on met à jour \texttt{size[u]}, \texttt{size[v]}.

\lstinputlisting{Scripts/UF2.py}


\section{$FIND$ : Compresser les chemins}

Au moment de chercher le représentant d'un élément on compresse le chemin

\lstinputlisting{Scripts/UF3.py}

